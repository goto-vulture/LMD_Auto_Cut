\documentclass[a3paper, 11pt, svgnames]{article}

\usepackage[margin=0.10cm]{geometry}
\usepackage[utf8]{inputenc}
\usepackage{tikz-uml}
\usepackage{float}

\newcommand{\userName}{User}
\newcommand{\guiName}{GUI}
\newcommand{\fileDialogName}{File dialog}
\newcommand{\loadDialogName}{Load dialog}
\newcommand{\calculationThreadName}{Calculation thread}
\newcommand{\outThreadName}{Thread output file}

\newcommand{\hintGreen}{green!75}
\newcommand{\hintRed}{red!75}



\begin{document}
    \begin{figure}[H]
        \centering
        \begin{tikzpicture}[scale=0.95]
            \begin{umlseqdiag}
                \umlactor[no ddots]{\userName}
                \umlobject[x=7, no ddots]{\guiName}
                %\umlobject[x=7, no ddots]{\fileDialogName}
                %\umlobject[no ddots]{\loadDialogName}

                \begin{umlfragment}[type=alt, label=On Windows, inner xsep=11]
                    \begin{umlcall}[op=Start\_Windows.bat, dt=10]{\userName}{\guiName}
                    \end{umlcall}
                    \umlfpart[On Linux]
                    \begin{umlcall}[op=Start\_Linux.sh]{\userName}{\guiName}
                    \end{umlcall}
                \end{umlfragment}

                \begin{umlcall}[op=File$\rightarrow$Open picture 1/2]{\userName}{\guiName}
                    \umlcreatecall[x=11, no ddots, dt=-1]{\guiName}{\fileDialogName}
                \end{umlcall}
                \begin{umlcall}[dt=2, op=Select picture 1/2]{\userName}{\fileDialogName}
                    \begin{umlcall}[dt=5, op=Get file name, return=file name]{\guiName}{\fileDialogName}
                        % Has no meaning here! This call will only create  some space between the previous call and the following return
                        \begin{umlcall}[type=return, draw=white, dt=-4]{\fileDialogName}{\guiName}
                        \end{umlcall}
                    \end{umlcall}
                    \begin{umlcall}[op=destroy, text=\hintRed]{\guiName}{\fileDialogName}
                    \end{umlcall}
                \end{umlcall}

                \umlcreatecall[no ddots]{\guiName}{\loadDialogName}
                \begin{umlcall}[op=load picture, return=numpy array]{\guiName}{\loadDialogName}
                    % Has no meaning here! This call will only create  some space between the previous call and the following return
                    \begin{umlcall}[type=return, draw=white, dt=-4]{\loadDialogName}{\guiName}
                    \end{umlcall}
                \end{umlcall}
                \begin{umlcall}[op=destroy, text=\hintRed]{\guiName}{\loadDialogName}
                \end{umlcall}

                % Adjust the position of the next call for the GUI
                \umlsdnode[dt=3]{\guiName}

                \begin{umlfragment}[type=GUI adjustments]
                    \begin{umlcallself}[op=Show pic on GUI]{\guiName}
                    \end{umlcallself}
                    \begin{umlcallself}[op=Activate RB ``Picture 1'' / ``Picture 2'']{\guiName}
                    \end{umlcallself}
                    \begin{umlcallself}[op=Select RB ``Picture 1'' / ``Picture 2'']{\guiName}
                    \end{umlcallself}
                    \begin{umlcallself}[op=Activate RB ``Same as picture 1 path'' / ``Same as pictur 2'']{\guiName}
                    \end{umlcallself}
                    \begin{umlcallself}[op=Select RB ``Same as picture 1 path'' / ``Same as picture 2 path'']{\guiName}
                    \end{umlcallself}
                    \begin{umlcallself}[op=Activate button ``Start calculation'']{\guiName}
                    \end{umlcallself}
                \end{umlfragment}

                % Adjust the position of the next call from user to GUI
                \umlsdnode[dt=82]{\userName}

                \begin{umlfragment}[type=loop, label={i: 0 - 3}, inner xsep=4]
                    \begin{umlcall}[op=Select cal points]{\userName}{\guiName}
                        \begin{umlcallself}[op=Draw cal point on pic]{\guiName}
                        \end{umlcallself}
                        \begin{umlcallself}[op=Update cal points, dt=-1]{\guiName}
                        \end{umlcallself}
                    \end{umlcall}
                \end{umlfragment}

                % Adjust the position of the user object before the next UML fragment will be created
                \umlsdnode[dt=5]{\userName}

                \begin{umlfragment}[name=readOutput]
                    \begin{umlcall}[op=Using ``Start calculation'' button, return=XML file]{\userName}{\guiName}
                        \umlcreatecall[no ddots]{\guiName}{\calculationThreadName}
                        \umlcreatecall[x=23, no ddots]{\guiName}{\outThreadName}
                        \begin{umlcall}[op=start, type=asynchron]{\guiName}{\calculationThreadName}
                            \begin{umlcall}[op=Write stdout + stderr, type=asynchron]{\calculationThreadName}{\outThreadName}
                            \end{umlcall}
                        \end{umlcall}
                        \umlsdnode[dt=2]{\outThreadName}
                        \umlsdnode[dt=7]{\calculationThreadName}

                        \begin{umlfragment}[type=loop]
                            \begin{umlcall}[op=Read last output line, type=asynchron, return=Last output line]{\guiName}{\outThreadName}
                            \end{umlcall}
                        \end{umlfragment}

                        \begin{umlcallself}[op={\texttt{\_\_label\_Connected\_Regions}}]{\calculationThreadName}
                        \end{umlcallself}
                        \begin{umlcallself}[op=Remove too small areas]{\calculationThreadName}
                        \end{umlcallself}
                        \begin{umlcallself}[op={\texttt{\_\_create\_Sub\_Pictures}}]{\calculationThreadName}
                        \end{umlcallself}
                        \begin{umlcallself}[op=Create shape collection]{\calculationThreadName}
                        \end{umlcallself}
                        \begin{umlcallself}[op=Save shape collection]{\calculationThreadName}
                        \end{umlcallself}
                     \end{umlcall}
                     \begin{umlcall}[op=destroy, text=\hintRed]{\guiName}{\calculationThreadName}
                     \end{umlcall}
                     \begin{umlcall}[op=destroy, text=\hintRed]{\guiName}{\outThreadName}
                     \end{umlcall}
                \end{umlfragment}

                % Hints
                \umlnote[x=24, y=-23]{readOutput}{In practice two separate files and threads for stdout and stderr are used}
            \end{umlseqdiag}
        \end{tikzpicture}
    \end{figure}
\end{document}


% Siehe: https://perso.ensta-paris.fr/~kielbasi/tikzuml/var/files/html/web-tikz-uml-userguidech5.html#x7-870005.4
% 5.5 To change preferences
%
% Thanks to the tikzumlset command, you can set colors for calls, fragments and objects:
%
% text:
%     allows to set the default text color (=black by default),
% draw:
%     allows to set the default color of edges and arrows (=black by default),
% fill object:
%     allows to set the default background color of objects (=yellow !20 by default),
% fill call:
%     allows to set the default background color for calls (=white by default),
% fill fragment:
%     allows to set the default background color for fragments (=white by default),
% font:
%     allows to set the default font style (=\small by default),
% object stereo:
%     allows to set the default font style (=object by default),
% call dt:
%     allows to set the default font style (=auto by default),
% call padding:
%     allows to set the default font style (=2 by default),
% call type:
%     allows to set the default font style (=synchron by default),
% fragment type:
%     allows to set the default font style (=opt by default),
% fragment inner xsep:
%     allows to set the default font style (=1 by default),
% fragment inner ysep:
%     allows to set the default font style (=1 by default),
% create call dt:
%     allows to set the default font style (=4 by default)
%
% You can also use the options text, draw and fill on a particular element, as in the example of introduction.
